%!TEX TS-program = xelatex
%!TEX encoding = UTF-8 Unicode
\documentclass[12pt]{amsart}
\usepackage[usenames,dvipsnames]{color}
\usepackage[colorlinks=true,citecolor=red,linkcolor=blue]{hyperref}
\usepackage[all]{hypcap}



\setlength{\textwidth}{6.2in}
\setlength{\textheight}{9in}
\setlength{\topmargin}{0in}
\setlength{\oddsidemargin}{0in}
\setlength{\evensidemargin}{0in}

%%%%%Babylonian cuneiform
\usepackage{pgffor, babyloniannum}
\newcommand{\Bnum}[1]{{\LARGE \babyloniannum{#1}}}

\begin{document}
\title{Babylonian cuneiform}
\author{Dave Richeson}

\maketitle

To create \href{http://en.wikipedia.org/wiki/Babylonian_numerals}{Babylonian cuneiform numerals} you must download the babyloniannum package at 

\href{http://www.ctan.org/tex-archive/macros/xetex/latex/babyloniannum}{http://www.ctan.org/tex-archive/macros/xetex/latex/babyloniannum}

You must also download and install the Santakku font from 

\href{http://www.hethport.uni-wuerzburg.de/cuneifont}{http://www.hethport.uni-wuerzburg.de/cuneifont}

Recall that the numbers 1 through 59 are written additively using the symbols 

\begin{center}
\Bnum{1}=1 and\Bnum{10}=10.
\end{center}

For example 47 is written 

\begin{center}
\Bnum{47}
\end{center}

This package is extremely easy to use. To obtain the symbols above type

\verb#\Bnum{47}#

Larger numbers are written in base 60, and this package will do the decomposition for you. For example, 

\[1234567890=1 \cdot 60^{5}+35 \cdot 60^{4}+15 \cdot 60^{3}+35 \cdot 60^{2}+31 \cdot 60^{1}+30 \cdot 60^{0}.\]

and 

\verb#\Bnum{1234567890}#

produces 

\begin{center}
\Bnum{1234567890}
\end{center}

\end{document}
