%%%%%%%%%%%%%%%%%%%%%%%%%%%%%%%%%%%%%%%%%%%%%%%%%%%%%%%%%%%%%%%%%%%%%%%%%%%%%%%%%%%%%%%
%%%%%%%%%%%%%%%%%%%%%%%%%%%%%%%%%%%%%%%%%%%%%%%%%%%%%%%%%%%%%%%%%%%%%%%%%%%%%%%%%%%%%%%
% 
% This top part of the document is called the 'preamble'.  Modify it with caution!
%
% The real document starts below where it says 'The main document starts here'.


\documentclass[12pt]{article}

\usepackage{amssymb,amsmath,amsthm}
\usepackage[top=1in, bottom=1in, left=1.25in, right=1.25in]{geometry}
\usepackage{fancyhdr}
\usepackage{listings}
\usepackage{enumerate}
\usepackage{hieroglf}
\usepackage{oands}
\usepackage{arevmath}
\usepackage{relsize}
\usepackage{times,txfonts}
\usepackage{graphicx}
\usepackage{float}

\newtheoremstyle{homework}% name of the style to be used
  {18pt}% measure of space to leave above the theorem. E.g.: 3pt
  {12pt}% measure of space to leave below the theorem. E.g.: 3pt
  {}% name of font to use in the body of the theorem
  {}% measure of space to indent
  {\bfseries}% name of head font
  {:}% punctuation between head and body
  {2ex}% space after theorem head; " " = normal interword space
  {}% Manually specify head
\theoremstyle{homework} 

% Set up an Exercise environment and a Solution label.
\newtheorem*{exercisecore}{Exercise \@currentlabel}
\newenvironment{exercise}[1]
{\def\@currentlabel{#1}\exercisecore}
{\endexercisecore}

\newcommand{\localhead}[1]{\par\smallskip\noindent\textbf{#1}\nobreak\\}%
\newcommand\solution{\localhead{Solution:}}

%%%%%%%%%%%%%%%%%%%%%%%%%%%%%%%%%%%%%%%%%%%%%%%%%%%%%%%%%%%%%%%%%%%%%%%%
%
% Stuff for getting the name/document date/title across the header
\makeatletter
\RequirePackage{fancyhdr}
\pagestyle{fancy}
\fancyfoot[C]{\ifnum \value{page} > 1\relax\thepage\fi}
\fancyhead[L]{\ifx\@doclabel\@empty\else\@doclabel\fi}
\fancyhead[C]{\ifx\@docdate\@empty\else\@docdate\fi}
\fancyhead[R]{\ifx\@docauthor\@empty\else\@docauthor\fi}
\headheight 15pt

\def\doclabel#1{\gdef\@doclabel{#1}}
\doclabel{Use {\tt\textbackslash doclabel\{MY LABEL\}}.}
\def\docdate#1{\gdef\@docdate{#1}}
\docdate{Use {\tt\textbackslash docdate\{MY DATE\}}.}
\def\docauthor#1{\gdef\@docauthor{#1}}
\docauthor{Use {\tt\textbackslash docauthor\{MY NAME\}}.}
\makeatother

% Shortcuts for blackboard bold number sets (reals, integers, etc.)
\newcommand{\Reals}{\ensuremath{\mathbb R}}
\newcommand{\Nats}{\ensuremath{\mathbb N}}
\newcommand{\Ints}{\ensuremath{\mathbb Z}}
\newcommand{\Rats}{\ensuremath{\mathbb Q}}
\newcommand{\Cplx}{\ensuremath{\mathbb C}}
%% Some equivalents that some people may prefer.
\let\RR\Reals
\let\NN\Nats
\let\II\Ints
\let\CC\Cplx

%%%%%%%%%%%%%%%%%%%%%%%%%%%%%%%%%%%%%%%%%%%%%%%%%%%%%%%%%%%%%%%%%%%%%%%%%%%%%%%%%%%%%%%
%%%%%%%%%%%%%%%%%%%%%%%%%%%%%%%%%%%%%%%%%%%%%%%%%%%%%%%%%%%%%%%%%%%%%%%%%%%%%%%%%%%%%%%
% 
% The main document start here.

% The following commands set up the material that appears in the header.
\doclabel{Math 316: HW 7}
\docauthor{Stefano Fochesatto}
\docdate{\today}

\begin{document}


\textbf{Section 6.2}

\begin{exercise}{2} A merchant doing business in Lucca doubled his money there and then spent 12 denarii. 
  On leaving, he went to Florence, where he also doubled his money and spent 12 denarii. 
  Returning home to Pisa, he there doubled his money and again spent 12 denarii, nothing remaining. 
  How much did he have in the beginning?\\
  \solution First let's convert the word problem into an equation with our modern notation,
  \begin{equation*}
    2(2(2x - 12)-12) = 0.
  \end{equation*}
  Expanding the equation,
  \begin{align*}
    2(2(2x - 12)-12) &= 0,\\
    2(4x - 36) &= 0,\\
    8x - 72 &= 0,\\
    8x &= 72,\\
    x &= 9.
  \end{align*}

\end{exercise}
\vspace{.5in}

\begin{exercise}{4} Given the squares of three successive odd numbers, show that the largest square exceeds
   the middle square by eight more than the middle square exceeds the smallest.\\
   \solution 
   First note that the three successive odd numbers have the form, 
   \begin{equation*}
     (x)^2, (x+2)^2, (x+4)^2.
   \end{equation*}
   Consider the middle square, 
   \begin{equation*}
     (x+2)^2 = x^2 + 4x + 4.
   \end{equation*}
   Thus the middle square exceeds the smallest by $4x + 4$. Now consider the largest square,
   \begin{align*}
    (x+4)^2 &= x^2 + 8x + 16,\\ 
            &= x^2 + 4x + 4x + 4 + 12,\\
            &= x^2 + 4x + 4 + 4x + 12,\\ 
            &= (x+2)^2 + 4x + 12.
   \end{align*}
   Thus the largest square exceeds the middle square by $4x + 12$. Now we can calculate the difference,
   \begin{equation*}
    4x + 12 - (4x + 4) = 8.
   \end{equation*}
   Thus the largest square exceeds the middle square by eight more than the middle square exceeds the smallest. 
   
\end{exercise}
\vspace{.5in}

\begin{exercise}{7} Fibonacci proved that if the sum of two consecutive integers is a square (that is if $n + (n-1) = u^2$ for some $u$), 
  then the square of the larger integer will equal the sum of two nonzero squares. Verify this 
  result and furnish several numerical examples.\\
  \solution Suppose that $n + (n-1) = u^2$ for some $u$. Through some algebra we get that, $2n - 1 = u^2$. Expanding $(n-1)^2$,
  \begin{align*}
    (n-1)^2 &= n^2 - 2n + 1,\\
    (n-1)^2 + 2n - 1 &= n^2,\\
    (n-1)^2 + u^2 &= n^2.
  \end{align*}
Thus we have shown that if the sum of two consecutive integers is square then the square of the larger integer is the sum of two nonzero squares. 
For and example consider $4,5$ note $4 + 5 = 3^2$ and $5^2 = 4^2 + 3^2$. Now consider $(12,13)$ note that $12 + 13 = 5^2$ and $13^2 = 5^2 + 12^2$.
Finally consider $(60,61)$, note that $60 + 61 = 11^2$ and $61^2 = 11^2 + 60^2$.
\end{exercise}
\vspace{.5in}



\begin{exercise}{17} Solve the following equation for $x$ and $y:$.
  \begin{equation*}
    a/x = y/b,
  \end{equation*} 
  \begin{equation*}
    x/y = c.
  \end{equation*} 
  For example, if the first and fourth term of a given ratio are 18 and 2, and the ratio of the second and third equals 4, 
  what are the second and third term.\\
  \solution Solving the second equation for $x$,
  \begin{equation*}
    x = cy.
  \end{equation*}
  Substituting into the first equation,
  \begin{align*}
    y/b &= a/(cy),\\
    y^2/b &= a/c,\\
    y^2 &= ab/c,\\
    y &= \sqrt{(ab)/c}.\\
  \end{align*}
  Now substituting back into the second equation we get,
  \begin{equation*}
    x = cy = c(\sqrt{(ab)/c}).
  \end{equation*}
\end{exercise}
\vspace{.5in}



\textbf{Section 6.3}

\begin{exercise}{2.a} Show that the sum of the first $n$ Fibonacci numbers with odd indices is 
  given by the formula, 
  \begin{equation*}
    F_1 + F_3 + F_5 + \dots + F_{2n - 1} = F{2n}.
  \end{equation*}
  \solution Consider $n = 1$, then we by definition get that, 
  \begin{equation*}
    F{2(1)} = F_{2} = 1 = F_{1}
  \end{equation*}
  Suppose that for some $n$,
  \begin{equation*}
    F_1 + F_3 + F_5 + \dots + F_{2n - 1} = F{2n}.
  \end{equation*}
  Now consider $F_{2(n+1)}$. By the definition of the Fibonacci sequence we know that,
  \begin{equation*}
    F_{2(n+1)} = F_{2n+2} = F_{2n + 1} + F_{2n}
  \end{equation*}
  Substituting the induction hypothesis,
  \begin{equation*}
    F_{2(n+1)} = F_{2n+2} = F_{2n + 1} + F_{2n - 1} + \dots + F_5 + F_3 + F_1.
  \end{equation*}
  \begin{equation*}
    F_{2(n+1)} = F_{2(n+1) - 1} + F_{2(n+1) - 2} + \dots + F_5 + F_3 + F_1.
  \end{equation*}
  Thus for all $n$ we know that,
  \begin{equation*}
    F_1 + F_3 + F_5 + \dots + F_{2n - 1} = F{2n}.
  \end{equation*}
\end{exercise}
\vspace{.5in}


\textbf{Section 7.1}

\begin{exercise}{1.a} Finds all three roots of each of the following cubic equations by first reducing them to cubics
  that lack a term in $x^2$.
  \begin{equation*}
    x^3 + 11x = 6x^2 + 6. 
  \end{equation*} 
  \solution To solve this equation using Cardano's method, let's first get it into standard form,
  \begin{equation*}
    x^3 -6x^2 + 11x - 6 = 0. 
  \end{equation*}
  Now we make the substitution, $x = y + 6/3$ which give us the following, 
\begin{align*}
  y^3 + (11 - \frac{(-6)^2}{3})y + (\frac{2(-6)^3}{27} - \frac{(-6)11}{3} - 6) &= 0\\
  y^3  - y + (0) &= 0\\
  y^3  - y &= 0\\
  (y)(y^2  - 1) &= 0
\end{align*}
So $y = 0,-1,1$ and solving for $x$ we get that $x = 2, 1, 3$.
\end{exercise}
\vspace{.5in}

\begin{exercise}{3.b} Using Cardan's formula obtain one root of the following cubic equation,
  \begin{equation*}
    x^3 + 15x = 6x^2 + 18
  \end{equation*}
  \solution First we find the equation in standard form,
  \begin{equation*}
    x^3 - 6x^2 + 15x - 18 = 0.
  \end{equation*}
  Now we apply the substitution, $x = y + 6/3$ to get rid of the $x^2$ term, 
  \begin{align*}
    y^3 + (15 - \frac{(-6)^2}{3})y + (\frac{2(-6)^3}{27} - \frac{(-6)15}{3} - 18) &= 0,\\
    y^3 + (3)y + (-4) &= 0,\\
    y^3 + 3y + &= 4.
  \end{align*}
 Now we know that $p = 3$ and $q = 4$ we can apply Cardan's formula to find a root $y$, 
 \begin{equation*}
   y = \sqrt[3]{\frac{4}{2} + \sqrt{\frac{4^2}{4} + \frac{3^3}{27}}} - \sqrt[3]{-\frac{4}{2} + \sqrt{\frac{4^2}{4} + \frac{3^3}{27}}} = 1
 \end{equation*}
 Thus we get a root $x = 3$.

\end{exercise}
\vspace{.5in}

\begin{exercise}{12} The following method of Viera is useful in solving the reduced cubic
  $x^3 + ax = b$. By substitution of $x = a/3y - y$, the given equation becomes $y^6 + by^3 - a^3/27 = 0$,
  a quadratic in $y^3$. By the quadratic formula, 
  \begin{equation*}
    y^3 = \frac{1}{2}(-b \pm \sqrt{b^2 + \frac{4a^3}{27}}).
  \end{equation*}
  from which $y$ and then $x$ can be determined. Use this method to find a root of the cubics $x^3 + 81x = 702$
  and $x^3 + 6x^2 + 18x + 13 = 0$.
  \solution Note that the first cubic is already in the required form, so applying the substitution $x = 81/3y - y$ yields,
  \begin{equation*}
    y^6 + 702y^3 - \frac{81^3}{27} = 0.
  \end{equation*}
  Applying the quadratic formula yields, 
  \begin{equation*}
    y^3 = \frac{1}{2}(-702 \pm \sqrt{(702)^2 + \frac{4(81)^3}{27}}) = 27,-729
  \end{equation*}
  Let $y = 3$ and finally substituting back into $x = 81/3y - y$ we get that $x = 6$. 

  \vspace{.25in}

  For the second equation let's start by substituting, $x = y - 6/3$ which gives us, 
  \begin{align*}
    y^3 + (18 - \frac{(6)^2}{3})y + (\frac{2(6)^3}{27} - \frac{(6)18}{3} + 13) &= 0,\\
    y^3 + 6y -7 &= 0,\\
    y^3 + 6y &= 7.
  \end{align*}
  Now applying the Viera substitution we know that $y = 6/3z - z$, and we get, 
  \begin{equation*}
    z^6 + 7z^3 - \frac{6^3}{27} = 0.
  \end{equation*}
  Now applying the quadratic formula we get that, 
  \begin{equation*}
    z^3 = \frac{1}{2}(-7 \pm \sqrt{(7)^2 + \frac{4(6)^3}{27}}) = 1, -8
  \end{equation*}
  Let $z = 1$ and by substitution into $y = 6/3z - z$ we get $y = 1$ and by another substitution into 
  $x = y - 6/3$ we finally get a root of $x = -1$
\end{exercise}
\vspace{.5in}



\textbf{Reflection}
\begin{enumerate}
  \item The last few problems felt really tedious and roundabout. Especially the last one, why would you use the Viera substitution when 
  you could just use Cardan's formula?
  \item I know I could write a better induction proof. 
\end{enumerate}





\end{document}