%%%%%%%%%%%%%%%%%%%%%%%%%%%%%%%%%%%%%%%%%%%%%%%%%%%%%%%%%%%%%%%%%%%%%%%%%%%%%%%%%%%%%%%
%%%%%%%%%%%%%%%%%%%%%%%%%%%%%%%%%%%%%%%%%%%%%%%%%%%%%%%%%%%%%%%%%%%%%%%%%%%%%%%%%%%%%%%
% 
% This top part of the document is called the 'preamble'.  Modify it with caution!
%
% The real document starts below where it says 'The main document starts here'.


\documentclass[12pt]{article}

\usepackage{amssymb,amsmath,amsthm}
\usepackage[top=1in, bottom=1in, left=1.25in, right=1.25in]{geometry}
\usepackage{fancyhdr}
\usepackage{listings}
\usepackage{enumerate}
\usepackage{hieroglf}
\usepackage{oands}
\usepackage{arevmath}
\usepackage{relsize}
\usepackage{times,txfonts}
\usepackage{graphicx}
\usepackage{float}

\newtheoremstyle{homework}% name of the style to be used
  {18pt}% measure of space to leave above the theorem. E.g.: 3pt
  {12pt}% measure of space to leave below the theorem. E.g.: 3pt
  {}% name of font to use in the body of the theorem
  {}% measure of space to indent
  {\bfseries}% name of head font
  {:}% punctuation between head and body
  {2ex}% space after theorem head; " " = normal interword space
  {}% Manually specify head
\theoremstyle{homework} 

% Set up an Exercise environment and a Solution label.
\newtheorem*{exercisecore}{Exercise \@currentlabel}
\newenvironment{exercise}[1]
{\def\@currentlabel{#1}\exercisecore}
{\endexercisecore}

\newcommand{\localhead}[1]{\par\smallskip\noindent\textbf{#1}\nobreak\\}%
\newcommand\solution{\localhead{Solution:}}

%%%%%%%%%%%%%%%%%%%%%%%%%%%%%%%%%%%%%%%%%%%%%%%%%%%%%%%%%%%%%%%%%%%%%%%%
%
% Stuff for getting the name/document date/title across the header
\makeatletter
\RequirePackage{fancyhdr}
\pagestyle{fancy}
\fancyfoot[C]{\ifnum \value{page} > 1\relax\thepage\fi}
\fancyhead[L]{\ifx\@doclabel\@empty\else\@doclabel\fi}
\fancyhead[C]{\ifx\@docdate\@empty\else\@docdate\fi}
\fancyhead[R]{\ifx\@docauthor\@empty\else\@docauthor\fi}
\headheight 15pt

\def\doclabel#1{\gdef\@doclabel{#1}}
\doclabel{Use {\tt\textbackslash doclabel\{MY LABEL\}}.}
\def\docdate#1{\gdef\@docdate{#1}}
\docdate{Use {\tt\textbackslash docdate\{MY DATE\}}.}
\def\docauthor#1{\gdef\@docauthor{#1}}
\docauthor{Use {\tt\textbackslash docauthor\{MY NAME\}}.}
\makeatother

% Shortcuts for blackboard bold number sets (reals, integers, etc.)
\newcommand{\Reals}{\ensuremath{\mathbb R}}
\newcommand{\Nats}{\ensuremath{\mathbb N}}
\newcommand{\Ints}{\ensuremath{\mathbb Z}}
\newcommand{\Rats}{\ensuremath{\mathbb Q}}
\newcommand{\Cplx}{\ensuremath{\mathbb C}}
%% Some equivalents that some people may prefer.
\let\RR\Reals
\let\NN\Nats
\let\II\Ints
\let\CC\Cplx

%%%%%%%%%%%%%%%%%%%%%%%%%%%%%%%%%%%%%%%%%%%%%%%%%%%%%%%%%%%%%%%%%%%%%%%%%%%%%%%%%%%%%%%
%%%%%%%%%%%%%%%%%%%%%%%%%%%%%%%%%%%%%%%%%%%%%%%%%%%%%%%%%%%%%%%%%%%%%%%%%%%%%%%%%%%%%%%
% 
% The main document start here.

% The following commands set up the material that appears in the header.
\doclabel{Math 316: HW 6}
\docauthor{Stefano Fochesatto}
\docdate{\today}

\begin{document}


\textbf{Section 5.3}

\begin{exercise}{21} Bhaskara, 1150. What number divided by 6 leaves a remainder of 5, divided by 5
  leaves a remainder of 4, divided by 4 leaves a remainder of 3, and divided by 3 leaves a remainder of 2? \\
  \begin{align*}
    N &\cong 5 mod 6,\\
    N &\cong 4 mod 5,\\
    N &\cong 3 mod 4,\\
    N &\cong 2 mod 3.
  \end{align*}
  \solution To solve this problem we must first compute the product $M$, of the moduli $m_i$,
  \begin{equation*}
    M = 6*5*4*3 = 360.
  \end{equation*}
  The next step involves computing all $M_i$ such that 
  \begin{equation*}
    M_i = \frac{M}{m_i}.
  \end{equation*}
  Therefore we get the following, 
\begin{align*}
  M_1 = &\frac{360}{6} = 60, \\
  M_2 = &\frac{360}{5} =72, \\
  M_3 = &\frac{360}{4} =90, \\
  M_4 = &\frac{360}{3} =120.
\end{align*}
Now we reduce each $M_i$ mod $m_i$, so find a $P_i$ such that, 
\begin{equation*}
  M_i = P_i \mod m_i.
\end{equation*}
Doing that we get,
\begin{align*}
  60 &\equiv 0 \mod 6.\\
  72 &\equiv 2 \mod 5.\\
  90 &\equiv 2 \mod 4.\\
  120 &\equiv 0 \mod 3.
\end{align*}
Now we need to find one for each $P_i$ so solving for some $x_i$ that gives, 
\begin{equation*}
  P_ix_i \equiv 1 \mod m_i.
\end{equation*}
 Doing this we get, 
\begin{align*}
 (0)(1) &\equiv 1 \mod 6\\
 (2)(4) &\equiv 1 \mod 5\\
 (2)(5) &\equiv 1 \mod 4\\
 (0)(1) &\equiv 1 \mod 3
\end{align*}
\end{exercise}
\vspace{.5in}


\textbf{Section 5.5}

\begin{exercise}{1} Solve the following quadratic equqitons with the araic method of complete the square.\\
  \begin{enumerate}
    \item $x^2 + 12x = 64$\\
    \solution First note that this is a type 4 problem with the form $ax^2 + bx = 2$. We complete the square by noting that, 
    \begin{equation*}
      (x + 6)^2 = x^2 + 12x + 36. 
    \end{equation*}
    So adding 36 to both sides we get that,
    \begin{align*}
      x^2 + 12x + 36 &= 64 + 36,\\
      (x + 6)^2 &= 100,\\
      (x + 6)^2 &= 10^2.
    \end{align*}
    Therefore $x = 4, -12$. 
    \vspace{.25in}

    \item $3x^2 + 10x = 32$\\
    \solution From the hint lets multiply both sides of the equation by 3 and and simplify the form of our equantio with a substitution of $y = 3x$, 
    \begin{align*}
      3x^2 + 10x &= 32,\\
      3(3x^2 + 10x) &= 3(32),\\
      9x^2 + 30x &= 96,\\
      (3x)^2 + 10(3x) &= 96,\\
      (y)^2 + 10(y) &= 96.
    \end{align*}
    Now our problem is a type 4 problem with the form $ax^2 + bx = 2$. We complete the square by noting that, 
    \begin{equation*}
      (y + 5)^2 = y^2 + 10y + 25. 
    \end{equation*}
    So adding 25 to both sides, 
    \begin{align*}
      y^2 + 10y + 25 &= 96 + 25,\\
      (y + 5)^2 &= 121,\\
      (y + 5)^2 &= (11)^2.
    \end{align*}
    Thus we get that $y = 6, -16$ and since $y = 3x$ we get that $x = 2, \frac{-16}{3}$
    \vspace{.25in}


  \end{enumerate}
\end{exercise}
\vspace{.5in}

\begin{exercise}{7}
  \begin{enumerate}
    \item Show that the cubic equation $x^3 + b^2c = b^2x$ can be solved by finding the intersection of the 
    parabola $x^2 = by$ and the hyperbola $y^2 + cx = x^2$.\\

    \solution We can show that the intersection of $x^2 = by$ and $y^2 + cx = x^2$ gives $x^3 + b^2c = b^2x$ through algebra.
    First solve the first equation for $y$, 
    \begin{equation*}
      y = \frac{x^2}{b}. 
    \end{equation*}
    Now substituting into the second equation and doing some algebra to get the third equation. 
    \begin{align*}
      (\frac{x^2}{b})^2 + cx &= x^2,\\
      \frac{x^4}{b^2} + cx &= x^2,\\
      \frac{x^4}{b^2} + cx &= x^2,\\
      \frac{x^3}{b^2} + c &= x,\\
      x^3 + b^2c &= b^2x.\\
    \end{align*}
    Therefore where the two conic sections intersect we get the solution to the cubic. 
    \vspace{.25in}


    \item Show that the cubic equation $x^3 + c = ax^2$ can be solved by finding the intersection of the parabola $y^2 + cx = ac$
    and the rectangular hyperbola $xy = c$. \\

    \solution Again we can show this through algebra. Solving the rectangular hyperbola for $y$,
    \begin{equation*}
      y = \frac{c}{x}.
    \end{equation*}
    Substotutinog into the parabola and doing some algebra to get the cubic,
    \begin{align*}
      y^2 + cx &= ac,\\
      (\frac{c}{x})^2 + cx &= ac,\\
      \frac{c^2}{x^2} + cx &= ac,\\
      c^2 + cx^3 &= acx^2,\\
      c + x^3 &= ax^2,\\
      x^3 + c &= ax^2.
    \end{align*}
    Thus where the two conic sections intersect we get the solution to the cubic. 
  \end{enumerate}
\end{exercise}
\vspace{.5in}

\textbf{Additional Problems}

\begin{exercise}{1}
\end{exercise}
\vspace{.5in}


\begin{exercise}{2}
\end{exercise}
\vspace{.5in}


\textbf{Reflection}
\begin{enumerate}
  \item 
  
  \item 

\end{enumerate}








\end{document}