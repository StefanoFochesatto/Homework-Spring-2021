%%%%%%%%%%%%%%%%%%%%%%%%%%%%%%%%%%%%%%%%%%%%%%%%%%%%%%%%%%%%%%%%%%%%%%%%%%%%%%%%%%%%%%%
%%%%%%%%%%%%%%%%%%%%%%%%%%%%%%%%%%%%%%%%%%%%%%%%%%%%%%%%%%%%%%%%%%%%%%%%%%%%%%%%%%%%%%%
% 
% This top part of the document is called the 'preamble'.  Modify it with caution!
%
% The real document starts below where it says 'The main document starts here'.

\documentclass[12pt]{article}

\usepackage{amssymb,amsmath,amsthm}
\usepackage[top=1in, bottom=1in, left=1.25in, right=1.25in]{geometry}
\usepackage{fancyhdr}
\usepackage{enumerate}
\usepackage{hieroglf}
\usepackage{times,txfonts}
\usepackage{graphicx}
\usepackage{float}

% Comment the following line to use TeX's default font of Computer Modern.
\usepackage{times,txfonts}

\newtheoremstyle{homework}% name of the style to be used
  {18pt}% measure of space to leave above the theorem. E.g.: 3pt
  {12pt}% measure of space to leave below the theorem. E.g.: 3pt
  {}% name of font to use in the body of the theorem
  {}% measure of space to indent
  {\bfseries}% name of head font
  {:}% punctuation between head and body
  {2ex}% space after theorem head; " " = normal interword space
  {}% Manually specify head
\theoremstyle{homework} 

% Set up an Exercise environment and a Solution label.
\newtheorem*{exercisecore}{Exercise \@currentlabel}
\newenvironment{exercise}[1]
{\def\@currentlabel{#1}\exercisecore}
{\endexercisecore}

\newcommand{\localhead}[1]{\par\smallskip\noindent\textbf{#1}\nobreak\\}%
\newcommand\solution{\localhead{Solution:}}

%%%%%%%%%%%%%%%%%%%%%%%%%%%%%%%%%%%%%%%%%%%%%%%%%%%%%%%%%%%%%%%%%%%%%%%%
%
% Stuff for getting the name/document date/title across the header
\makeatletter
\RequirePackage{fancyhdr}
\pagestyle{fancy}
\fancyfoot[C]{\ifnum \value{page} > 1\relax\thepage\fi}
\fancyhead[L]{\ifx\@doclabel\@empty\else\@doclabel\fi}
\fancyhead[C]{\ifx\@docdate\@empty\else\@docdate\fi}
\fancyhead[R]{\ifx\@docauthor\@empty\else\@docauthor\fi}
\headheight 15pt

\def\doclabel#1{\gdef\@doclabel{#1}}
\doclabel{Use {\tt\textbackslash doclabel\{MY LABEL\}}.}
\def\docdate#1{\gdef\@docdate{#1}}
\docdate{Use {\tt\textbackslash docdate\{MY DATE\}}.}
\def\docauthor#1{\gdef\@docauthor{#1}}
\docauthor{Use {\tt\textbackslash docauthor\{MY NAME\}}.}
\makeatother

% Shortcuts for blackboard bold number sets (reals, integers, etc.)
\newcommand{\Reals}{\ensuremath{\mathbb R}}
\newcommand{\IRats}{\ensuremath{\mathbb I}}
\newcommand{\Nats}{\ensuremath{\mathbb N}}
\newcommand{\Ints}{\ensuremath{\mathbb Z}}
\newcommand{\Rats}{\ensuremath{\mathbb Q}}
\newcommand{\Cplx}{\ensuremath{\mathbb C}}
%% Some equivalents that some people may prefer.
\let\RR\Reals
\let\NN\Nats
\let\II\Ints
\let\CC\Cplx


\doclabel{Math F316}
\docauthor{Stefano Fochesatto}
\docdate{\today}% DATE READING SENTENCES ARE DUE GOES HERE


%%%% Main document starts here.

\begin{document}

\section*{5.1-5.2}

\begin{description}
\item[Enlightening summary \#1:] Section 5.1 is a history heavy chapter. It begins by recounting the fall of Julius Caesar and the Greek empire, then discusses the
history of the Roman Empire and it's own religious transformation. The text describes Roman Egypt as "a sad record of short sighted exploitation by an absentee landlord",
which also aptly describes the sentiment around higher learning. The chief Roman concern being the construction of impressive engineering projects instead of mathematical discovery. 
The text describes the spread of christianity as a period of growing antirationalism. We also saw the eventual Byzantine empire become a refuge for Greek learning in the form of copyist who preserved literary text. 

   


\item[Enlightening summary \#2:] 
Section 5.2 is devoted to the Greek's emancipation of Algebra in the form of Diophantus' Arithmetica. Described as the last great 
mathematician of classical antiquity, Diophantus' chief contribution comes in the form of the Arithmetica and the "syncopated algebra" shorthand
used to express problems. The Arithmetica was similar to the Rhind Papyrus, an assortment of problems and their solutions. The 
syncopated algebra shorthand had many problems, with no addition symbol negative terms had to be grouped together, there was no zero quantity and the concept of 
negative solutions was absurd.  


\item[Interesting:]                                                                                        
 The text claims that the beginning of Roman rule brought peace and tranquility to Alexandria, and there is an interesting quoted from Edward Gibbon which empahsises that point. 
 I thought it was interesting that despite all the peace and tranquility there was still "no inclination or aptitude for theoretical studies". 
 Usually periods of peace and prosperity give rise to a leisure class that has the time and resources to advance theoretical studies. 

 
\item[Confusing:] The text describes no order of operations system in regards to the "syncopated algebra" shorthand. I'm confused
as to how any of the quantities with parenthesis were depicted, and especially when those quantities are raised to an exponent.   
 


 
\end{description}
\end{document}
