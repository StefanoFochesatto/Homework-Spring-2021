%%%%%%%%%%%%%%%%%%%%%%%%%%%%%%%%%%%%%%%%%%%%%%%%%%%%%%%%%%%%%%%%%%%%%%%%%%%%%%%%%%%%%%%
%%%%%%%%%%%%%%%%%%%%%%%%%%%%%%%%%%%%%%%%%%%%%%%%%%%%%%%%%%%%%%%%%%%%%%%%%%%%%%%%%%%%%%%
% 
% This top part of the document is called the 'preamble'.  Modify it with caution!
%
% The real document starts below where it says 'The main document starts here'.

\documentclass[12pt]{article}

\usepackage{amssymb,amsmath,amsthm}
\usepackage[top=1in, bottom=1in, left=1.25in, right=1.25in]{geometry}
\usepackage{fancyhdr}
\usepackage{enumerate}
\usepackage{hieroglf}

% Comment the following line to use TeX's default font of Computer Modern.
\usepackage{times,txfonts}

\newtheoremstyle{homework}% name of the style to be used
  {18pt}% measure of space to leave above the theorem. E.g.: 3pt
  {12pt}% measure of space to leave below the theorem. E.g.: 3pt
  {}% name of font to use in the body of the theorem
  {}% measure of space to indent
  {\bfseries}% name of head font
  {:}% punctuation between head and body
  {2ex}% space after theorem head; " " = normal interword space
  {}% Manually specify head
\theoremstyle{homework} 

% Set up an Exercise environment and a Solution label.
\newtheorem*{exercisecore}{Exercise \@currentlabel}
\newenvironment{exercise}[1]
{\def\@currentlabel{#1}\exercisecore}
{\endexercisecore}

\newcommand{\localhead}[1]{\par\smallskip\noindent\textbf{#1}\nobreak\\}%
\newcommand\solution{\localhead{Solution:}}

%%%%%%%%%%%%%%%%%%%%%%%%%%%%%%%%%%%%%%%%%%%%%%%%%%%%%%%%%%%%%%%%%%%%%%%%
%
% Stuff for getting the name/document date/title across the header
\makeatletter
\RequirePackage{fancyhdr}
\pagestyle{fancy}
\fancyfoot[C]{\ifnum \value{page} > 1\relax\thepage\fi}
\fancyhead[L]{\ifx\@doclabel\@empty\else\@doclabel\fi}
\fancyhead[C]{\ifx\@docdate\@empty\else\@docdate\fi}
\fancyhead[R]{\ifx\@docauthor\@empty\else\@docauthor\fi}
\headheight 15pt

\def\doclabel#1{\gdef\@doclabel{#1}}
\doclabel{Use {\tt\textbackslash doclabel\{MY LABEL\}}.}
\def\docdate#1{\gdef\@docdate{#1}}
\docdate{Use {\tt\textbackslash docdate\{MY DATE\}}.}
\def\docauthor#1{\gdef\@docauthor{#1}}
\docauthor{Use {\tt\textbackslash docauthor\{MY NAME\}}.}
\makeatother

% Shortcuts for blackboard bold number sets (reals, integers, etc.)
\newcommand{\Reals}{\ensuremath{\mathbb R}}
\newcommand{\IRats}{\ensuremath{\mathbb I}}
\newcommand{\Nats}{\ensuremath{\mathbb N}}
\newcommand{\Ints}{\ensuremath{\mathbb Z}}
\newcommand{\Rats}{\ensuremath{\mathbb Q}}
\newcommand{\Cplx}{\ensuremath{\mathbb C}}
%% Some equivalents that some people may prefer.
\let\RR\Reals
\let\NN\Nats
\let\II\Ints
\let\CC\Cplx


\doclabel{Math F316}
\docauthor{Stefano Fochesatto}
\docdate{\today}% DATE READING SENTENCES ARE DUE GOES HERE


%%%% Main document starts here.

\begin{document}

\section*{3.3-3.5}

\begin{description}
    %applied arithmetic and egyptian math
\item[Enlightening summary \#1:] Continuing from the previous reading, Section 3.4 introduces us to 
Hippocrates and the three problems of antiquity. The text recounts Hippocrates' misfortune and his later success,
being one of the first to support himself from teaching mathematics. When Hippocrates arrived in Athens at the beginning of his 
math/philosophy career the three problems of antiquity were at the attention of all geometers. The rest of the section discusses Hippocrates' attempts at these problems, 
making significant progress on the duplication of the cube, and the squared circle. 
 
\item[Enlightening summary \#2:] Section 3.5 starts with Hippias and the story of the sophists. The sophists where a group of traveling lecturers who were said to teach 
students logic, and debate. They had many criticisms, to some the sophists taught people to be con artists, training cleverness rather than virtue. Hippias, the most successful
among them was characterized in Plato's writing as being an arrogant, boastful, eccentric man with a wide breadth of knowledge and exceptional memory. What cements Hippias' legacy is his
invention of the quadratrix. The construction of the quadratrix is given in the text, but to summarize, the quadratrix essentially provides a relationship between the length of an edge and any angle between 0 and 90 degrees. 
The rest of the section discusses Plato's academy, emphasizing his importance in the field of mathematics even though he was primarily a philosopher. 

\item[Interesting:] I thoroughly enjoyed the history in this section of reading. I thought that the section that talked about how the sophists taught cleverness instead of virtue
really characterized them as grifters and con artists. Especially when Hippias brags about his Sicilian lecture tour, and his refusal to teach the youth in Sparta. The text also describes
Plato's school as a religious brotherhood. It's unfortunate that at the time religious affiliation was the only way to attain legal recognition. Had there been some other way it's likely that the \textbf{christian} Justinian 
would have let them continue. I wonder if that was the only reason for their religious affiliation? And if not, how would worship play a role in Plato's school? 
 

\item[Confusing:] I wish the text would have elaborated on how Wantzel's algebraic criteria for the problems of antiquity were inspired and derived. Also the section that 
talks about Hippocrates' "lune" attempt at the squaring the circle problem had me very confused with the algebra. Looking back it's because I kept confusing the segment $AC$ with two separate variables $A$ and $C$. Probably because
I have my algebra hat on instead of my geometry hat.  


\end{description}
\end{document}
