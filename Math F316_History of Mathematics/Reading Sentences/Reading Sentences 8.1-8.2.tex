%%%%%%%%%%%%%%%%%%%%%%%%%%%%%%%%%%%%%%%%%%%%%%%%%%%%%%%%%%%%%%%%%%%%%%%%%%%%%%%%%%%%%%%
%%%%%%%%%%%%%%%%%%%%%%%%%%%%%%%%%%%%%%%%%%%%%%%%%%%%%%%%%%%%%%%%%%%%%%%%%%%%%%%%%%%%%%%
% 
% This top part of the document is called the 'preamble'.  Modify it with caution!
%
% The real document starts below where it says 'The main document starts here'.

\documentclass[12pt]{article}

\usepackage{amssymb,amsmath,amsthm}
\usepackage[top=1in, bottom=1in, left=1.25in, right=1.25in]{geometry}
\usepackage{fancyhdr}
\usepackage{enumerate}
\usepackage{hieroglf}
\usepackage{times,txfonts}
\usepackage{graphicx}
\usepackage{float}

% Comment the following line to use TeX's default font of Computer Modern.
\usepackage{times,txfonts}

\newtheoremstyle{homework}% name of the style to be used
  {18pt}% measure of space to leave above the theorem. E.g.: 3pt
  {12pt}% measure of space to leave below the theorem. E.g.: 3pt
  {}% name of font to use in the body of the theorem
  {}% measure of space to indent
  {\bfseries}% name of head font
  {:}% punctuation between head and body
  {2ex}% space after theorem head; " " = normal interword space
  {}% Manually specify head
\theoremstyle{homework} 

% Set up an Exercise environment and a Solution label.
\newtheorem*{exercisecore}{Exercise \@currentlabel}
\newenvironment{exercise}[1]
{\def\@currentlabel{#1}\exercisecore}
{\endexercisecore}

\newcommand{\localhead}[1]{\par\smallskip\noindent\textbf{#1}\nobreak\\}%
\newcommand\solution{\localhead{Solution:}}

%%%%%%%%%%%%%%%%%%%%%%%%%%%%%%%%%%%%%%%%%%%%%%%%%%%%%%%%%%%%%%%%%%%%%%%%
%
% Stuff for getting the name/document date/title across the header
\makeatletter
\RequirePackage{fancyhdr}
\pagestyle{fancy}
\fancyfoot[C]{\ifnum \value{page} > 1\relax\thepage\fi}
\fancyhead[L]{\ifx\@doclabel\@empty\else\@doclabel\fi}
\fancyhead[C]{\ifx\@docdate\@empty\else\@docdate\fi}
\fancyhead[R]{\ifx\@docauthor\@empty\else\@docauthor\fi}
\headheight 15pt

\def\doclabel#1{\gdef\@doclabel{#1}}
\doclabel{Use {\tt\textbackslash doclabel\{MY LABEL\}}.}
\def\docdate#1{\gdef\@docdate{#1}}
\docdate{Use {\tt\textbackslash docdate\{MY DATE\}}.}
\def\docauthor#1{\gdef\@docauthor{#1}}
\docauthor{Use {\tt\textbackslash docauthor\{MY NAME\}}.}
\makeatother

% Shortcuts for blackboard bold number sets (reals, integers, etc.)
\newcommand{\Reals}{\ensuremath{\mathbb R}}
\newcommand{\IRats}{\ensuremath{\mathbb I}}
\newcommand{\Nats}{\ensuremath{\mathbb N}}
\newcommand{\Ints}{\ensuremath{\mathbb Z}}
\newcommand{\Rats}{\ensuremath{\mathbb Q}}
\newcommand{\Cplx}{\ensuremath{\mathbb C}}
%% Some equivalents that some people may prefer.
\let\RR\Reals
\let\NN\Nats
\let\II\Ints
\let\CC\Cplx


\doclabel{Math F316}
\docauthor{Stefano Fochesatto}
\docdate{\today}% DATE READING SENTENCES ARE DUE GOES HERE


%%%% Main document starts here.

\begin{document}

\section*{8.1-8.2}

\begin{description}
\item[Enlightening summary \#1:] Chapter 8 recounts much of the modern mathematics that came about throughout the late 
fifteenth and sixteenth centuries. The chapter begins with the story of Galileo. Originally a medical student, Galileo would become an important martyr, spending most of his life disproving Aristotelian notions of astronomy, 
and enduring the many forms of punishment from dissenting establishments. Finally we begin to see bigger steps towards refining mathematical notation. 
Many people had a hand in developing the modern notation, Johann Widmann introduced the $+$, and $-$ symbols. Robert Recode introduced the $=$ 
symbol. Thomas Harriot introduced the $<$, and $>$ symbols. We also saw the $\div$ and $\sqrt{}$ symbols come about. Francois Vieta Introduced a symbolic notatino 
for variables, allowing for a more general handling of equation. Finally Rene Descartes introduced the $x^x$ power notation.
We saw Simon Stevin popularize the use of decimal fractions with \textit{The Tenth}. We also saw John Napier develop logarithms to aid in the 
computations in astronomy. Finally we saw Johannes Kepler and Tycho Brahe revolutionize astronomical mechanics. 
 


\item[Enlightening summary \#2:] The following section described the life of Rene Descartes. In \textit{La Geometrie} Descartes laid the foundation
for the beginnings of calculus and analytical geometry. Beyond that Descartes was a renowned philosopher, who strongly believed that the certainty of 
mathematics should serve as a model for other branches of study. The section ends with a piece on the discovery of Desargues theorem and the development of perspective geometry.



\item[Interesting:] I thought Kepler's quip about astrology was very funny.\\ 
\noindent\textit{Mother Astronomy would certainly starve if the daughter Astrology did not earn their bread.} 
I thought it was clever that typesetters would rotate the radical symbol for less than and greater than. 
 
\item[Confusing:]  I am still a little confused as to how astronomers actually used Napier's book to calculate large products. Did they 
sum the logs together, and then reverse search the sum to find the product?  

 
\end{description}
\end{document}
