%%%%%%%%%%%%%%%%%%%%%%%%%%%%%%%%%%%%%%%%%%%%%%%%%%%%%%%%%%%%%%%%%%%%%%%%%%%%%%%%%%%%%%%
%%%%%%%%%%%%%%%%%%%%%%%%%%%%%%%%%%%%%%%%%%%%%%%%%%%%%%%%%%%%%%%%%%%%%%%%%%%%%%%%%%%%%%%
% 
% This top part of the document is called the 'preamble'.  Modify it with caution!
%
% The real document starts below where it says 'The main document starts here'.

\documentclass[12pt]{article}

\usepackage{amssymb,amsmath,amsthm}
\usepackage[top=1in, bottom=1in, left=1.25in, right=1.25in]{geometry}
\usepackage{fancyhdr}
\usepackage{enumerate}
\usepackage{hieroglf}
\usepackage{times,txfonts}
\usepackage{graphicx}
\usepackage{float}

% Comment the following line to use TeX's default font of Computer Modern.
\usepackage{times,txfonts}

\newtheoremstyle{homework}% name of the style to be used
  {18pt}% measure of space to leave above the theorem. E.g.: 3pt
  {12pt}% measure of space to leave below the theorem. E.g.: 3pt
  {}% name of font to use in the body of the theorem
  {}% measure of space to indent
  {\bfseries}% name of head font
  {:}% punctuation between head and body
  {2ex}% space after theorem head; " " = normal interword space
  {}% Manually specify head
\theoremstyle{homework} 

% Set up an Exercise environment and a Solution label.
\newtheorem*{exercisecore}{Exercise \@currentlabel}
\newenvironment{exercise}[1]
{\def\@currentlabel{#1}\exercisecore}
{\endexercisecore}

\newcommand{\localhead}[1]{\par\smallskip\noindent\textbf{#1}\nobreak\\}%
\newcommand\solution{\localhead{Solution:}}

%%%%%%%%%%%%%%%%%%%%%%%%%%%%%%%%%%%%%%%%%%%%%%%%%%%%%%%%%%%%%%%%%%%%%%%%
%
% Stuff for getting the name/document date/title across the header
\makeatletter
\RequirePackage{fancyhdr}
\pagestyle{fancy}
\fancyfoot[C]{\ifnum \value{page} > 1\relax\thepage\fi}
\fancyhead[L]{\ifx\@doclabel\@empty\else\@doclabel\fi}
\fancyhead[C]{\ifx\@docdate\@empty\else\@docdate\fi}
\fancyhead[R]{\ifx\@docauthor\@empty\else\@docauthor\fi}
\headheight 15pt

\def\doclabel#1{\gdef\@doclabel{#1}}
\doclabel{Use {\tt\textbackslash doclabel\{MY LABEL\}}.}
\def\docdate#1{\gdef\@docdate{#1}}
\docdate{Use {\tt\textbackslash docdate\{MY DATE\}}.}
\def\docauthor#1{\gdef\@docauthor{#1}}
\docauthor{Use {\tt\textbackslash docauthor\{MY NAME\}}.}
\makeatother

% Shortcuts for blackboard bold number sets (reals, integers, etc.)
\newcommand{\Reals}{\ensuremath{\mathbb R}}
\newcommand{\IRats}{\ensuremath{\mathbb I}}
\newcommand{\Nats}{\ensuremath{\mathbb N}}
\newcommand{\Ints}{\ensuremath{\mathbb Z}}
\newcommand{\Rats}{\ensuremath{\mathbb Q}}
\newcommand{\Cplx}{\ensuremath{\mathbb C}}
%% Some equivalents that some people may prefer.
\let\RR\Reals
\let\NN\Nats
\let\II\Ints
\let\CC\Cplx


\doclabel{Math F316}
\docauthor{Stefano Fochesatto}
\docdate{\today}% DATE READING SENTENCES ARE DUE GOES HERE


%%%% Main document starts here.

\begin{document}

\section*{4.3-4.5}

\begin{description}
\item[Enlightening summary \#1:]
Section 4.3 picks up from last week's reading by providing an overview of books 7,8 and 9 from Euclid's Elements. Euclid began with defining several 
ideas such as divisibility, prime numbers, and composite numbers. We are introduced to the Euclidean algorithm, which is used to find the greatest common divisor between two numbers and
also find a linear combination representation. The rest of the chapter focuses on the numerous number theory propositions that come from primality, greatest common denominators and linear combinations. 
With some of the most important results being the fundamental theorem of arithmetic and the infinity of primes. 
 
\item[Enlightening summary \#2:] The last two sections of the reading focus on the many contributions from Eratosthenes and Archimedes. Beyond his contributions to
geography in the form of Geographica, Eratosthenes had a solution to Delian problem, and discovered the "sieve" method for finding prime numbers. To Plato's dismay,
Eratosthenes' solution requires more than a compass and straight edge. A contemporary to Eratosthenes, Archimedes was an incredibly skilled engineer and mathematician.
Archimedes approximated $\pi$ by a clever proof involving a regular polygon inscribed in a circle which is also inscribed in another regular polygon. Archimedes also used this similar idea and 
applied it to the quadrature of parabolic segments, by continually inscribing triangles under the curve. 

 


\item[Interesting:] I found Eratosthenes' Sieve super interesting. I'm actually familiar with it the algorithm, and have used it a few times for some coding projects but was unaware of the history behind it's
discoverer. Beyond that, I thought his calculation of the earth's circumference was incredibly creative. And of course all the stories about Archimedes were great, especially the passage about him launching 
a ship all by himself.    
 
 

\item[Confusing:] At first a lot of the number theory results confused me. It took me a while to resurface the familiarity that I had when I was taking abstract algebra. I would also be 
interested in see how Archimedes' method of quadrature performs against other methods of numerical integration. I'd imagine it performs comparably to Riemann, Trapezoid, or Simpsons rule quadrature. 


\end{description}
\end{document}
