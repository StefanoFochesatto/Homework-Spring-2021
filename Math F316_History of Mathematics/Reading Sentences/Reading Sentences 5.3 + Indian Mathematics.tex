%%%%%%%%%%%%%%%%%%%%%%%%%%%%%%%%%%%%%%%%%%%%%%%%%%%%%%%%%%%%%%%%%%%%%%%%%%%%%%%%%%%%%%%
%%%%%%%%%%%%%%%%%%%%%%%%%%%%%%%%%%%%%%%%%%%%%%%%%%%%%%%%%%%%%%%%%%%%%%%%%%%%%%%%%%%%%%%
% 
% This top part of the document is called the 'preamble'.  Modify it with caution!
%
% The real document starts below where it says 'The main document starts here'.

\documentclass[12pt]{article}

\usepackage{amssymb,amsmath,amsthm}
\usepackage[top=1in, bottom=1in, left=1.25in, right=1.25in]{geometry}
\usepackage{fancyhdr}
\usepackage{enumerate}
\usepackage{hieroglf}
\usepackage{times,txfonts}
\usepackage{graphicx}
\usepackage{float}

% Comment the following line to use TeX's default font of Computer Modern.
\usepackage{times,txfonts}

\newtheoremstyle{homework}% name of the style to be used
  {18pt}% measure of space to leave above the theorem. E.g.: 3pt
  {12pt}% measure of space to leave below the theorem. E.g.: 3pt
  {}% name of font to use in the body of the theorem
  {}% measure of space to indent
  {\bfseries}% name of head font
  {:}% punctuation between head and body
  {2ex}% space after theorem head; " " = normal interword space
  {}% Manually specify head
\theoremstyle{homework} 

% Set up an Exercise environment and a Solution label.
\newtheorem*{exercisecore}{Exercise \@currentlabel}
\newenvironment{exercise}[1]
{\def\@currentlabel{#1}\exercisecore}
{\endexercisecore}

\newcommand{\localhead}[1]{\par\smallskip\noindent\textbf{#1}\nobreak\\}%
\newcommand\solution{\localhead{Solution:}}

%%%%%%%%%%%%%%%%%%%%%%%%%%%%%%%%%%%%%%%%%%%%%%%%%%%%%%%%%%%%%%%%%%%%%%%%
%
% Stuff for getting the name/document date/title across the header
\makeatletter
\RequirePackage{fancyhdr}
\pagestyle{fancy}
\fancyfoot[C]{\ifnum \value{page} > 1\relax\thepage\fi}
\fancyhead[L]{\ifx\@doclabel\@empty\else\@doclabel\fi}
\fancyhead[C]{\ifx\@docdate\@empty\else\@docdate\fi}
\fancyhead[R]{\ifx\@docauthor\@empty\else\@docauthor\fi}
\headheight 15pt

\def\doclabel#1{\gdef\@doclabel{#1}}
\doclabel{Use {\tt\textbackslash doclabel\{MY LABEL\}}.}
\def\docdate#1{\gdef\@docdate{#1}}
\docdate{Use {\tt\textbackslash docdate\{MY DATE\}}.}
\def\docauthor#1{\gdef\@docauthor{#1}}
\docauthor{Use {\tt\textbackslash docauthor\{MY NAME\}}.}
\makeatother

% Shortcuts for blackboard bold number sets (reals, integers, etc.)
\newcommand{\Reals}{\ensuremath{\mathbb R}}
\newcommand{\IRats}{\ensuremath{\mathbb I}}
\newcommand{\Nats}{\ensuremath{\mathbb N}}
\newcommand{\Ints}{\ensuremath{\mathbb Z}}
\newcommand{\Rats}{\ensuremath{\mathbb Q}}
\newcommand{\Cplx}{\ensuremath{\mathbb C}}
%% Some equivalents that some people may prefer.
\let\RR\Reals
\let\NN\Nats
\let\II\Ints
\let\CC\Cplx


\doclabel{Math F316}
\docauthor{Stefano Fochesatto}
\docdate{\today}% DATE READING SENTENCES ARE DUE GOES HERE


%%%% Main document starts here.

\begin{document}

\section*{Indian Mathematics}

\begin{description}
\item[Enlightening summary \#1:] The section in our text provides a brief overview of the contributors to early Indian mathematics. We saw Aryabhata's approximation of $\pi$ as well as his introduction of negative numbers.  
The section also describes Brahmagupta's method for finding the general solution to $ax + by = c$, the proof in the text relies heavily on Euclid's Lemma. 
We are also introduced to Bhaskara and his work Siddhartha Siromani. We are shown an example from Bhaskara's Lilavati, which employs 
Euclid's algorithm to find the general solution of a linear diophantine equation.   


   
\item[Enlightening summary \#2:]  The additional reading goes further into detail on the discoveries made in early Indian mathematics, as well as the development of the Hindu-Arabic number system. The early Indian number system was heavily influenced by the Chinese counting board and Muslim Decimal fractions. The reading also discusses some geometric ideas. We see 
a Pythagorean Theorem construction that looks very similar to the Chinese 3,4,5 construction, as well as Brahmagupta's area of a cyclic quadrilateral. Their algebraic prowess is described as being able to solve quadratic equations and find integral solutions to linear indeterminate equations. There is a small section on combinatorics suggesting that they were able to calculate combinations and permutations, however, there is not much information on how they derived those rules. Finally, the section on trigonometry describes how they were able to interpolate values of trig functions, and how they even discovered the power series approximation for sin and cos. 




\item[Interesting:]  Their discovery of the sin and cos power series is very interesting and seems very ahead of their time. Also, the fact that \                                                                                        
Brahmagupta had an interpolation scheme that was similar to Newton's, almost a millennium before is really crazy. 

 

\item[Confusing:] I could not understand figure 6.2 at all. 


 
\end{description}
\end{document}
