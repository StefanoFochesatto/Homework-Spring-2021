%%%%%%%%%%%%%%%%%%%%%%%%%%%%%%%%%%%%%%%%%%%%%%%%%%%%%%%%%%%%%%%%%%%%%%%%%%%%%%%%%%%%%%%
%%%%%%%%%%%%%%%%%%%%%%%%%%%%%%%%%%%%%%%%%%%%%%%%%%%%%%%%%%%%%%%%%%%%%%%%%%%%%%%%%%%%%%%
% 
% This top part of the document is called the 'preamble'.  Modify it with caution!
%
% The real document starts below where it says 'The main document starts here'.

\documentclass[12pt]{article}

\usepackage{amssymb,amsmath,amsthm}
\usepackage[top=1in, bottom=1in, left=1.25in, right=1.25in]{geometry}
\usepackage{fancyhdr}
\usepackage{enumerate}
\usepackage{hieroglf}

% Comment the following line to use TeX's default font of Computer Modern.
\usepackage{times,txfonts}

\newtheoremstyle{homework}% name of the style to be used
  {18pt}% measure of space to leave above the theorem. E.g.: 3pt
  {12pt}% measure of space to leave below the theorem. E.g.: 3pt
  {}% name of font to use in the body of the theorem
  {}% measure of space to indent
  {\bfseries}% name of head font
  {:}% punctuation between head and body
  {2ex}% space after theorem head; " " = normal interword space
  {}% Manually specify head
\theoremstyle{homework} 

% Set up an Exercise environment and a Solution label.
\newtheorem*{exercisecore}{Exercise \@currentlabel}
\newenvironment{exercise}[1]
{\def\@currentlabel{#1}\exercisecore}
{\endexercisecore}

\newcommand{\localhead}[1]{\par\smallskip\noindent\textbf{#1}\nobreak\\}%
\newcommand\solution{\localhead{Solution:}}

%%%%%%%%%%%%%%%%%%%%%%%%%%%%%%%%%%%%%%%%%%%%%%%%%%%%%%%%%%%%%%%%%%%%%%%%
%
% Stuff for getting the name/document date/title across the header
\makeatletter
\RequirePackage{fancyhdr}
\pagestyle{fancy}
\fancyfoot[C]{\ifnum \value{page} > 1\relax\thepage\fi}
\fancyhead[L]{\ifx\@doclabel\@empty\else\@doclabel\fi}
\fancyhead[C]{\ifx\@docdate\@empty\else\@docdate\fi}
\fancyhead[R]{\ifx\@docauthor\@empty\else\@docauthor\fi}
\headheight 15pt

\def\doclabel#1{\gdef\@doclabel{#1}}
\doclabel{Use {\tt\textbackslash doclabel\{MY LABEL\}}.}
\def\docdate#1{\gdef\@docdate{#1}}
\docdate{Use {\tt\textbackslash docdate\{MY DATE\}}.}
\def\docauthor#1{\gdef\@docauthor{#1}}
\docauthor{Use {\tt\textbackslash docauthor\{MY NAME\}}.}
\makeatother

% Shortcuts for blackboard bold number sets (reals, integers, etc.)
\newcommand{\Reals}{\ensuremath{\mathbb R}}
\newcommand{\IRats}{\ensuremath{\mathbb I}}
\newcommand{\Nats}{\ensuremath{\mathbb N}}
\newcommand{\Ints}{\ensuremath{\mathbb Z}}
\newcommand{\Rats}{\ensuremath{\mathbb Q}}
\newcommand{\Cplx}{\ensuremath{\mathbb C}}
%% Some equivalents that some people may prefer.
\let\RR\Reals
\let\NN\Nats
\let\II\Ints
\let\CC\Cplx


\doclabel{Math F316}
\docauthor{Stefano Fochesatto}
\docdate{\today}% DATE READING SENTENCES ARE DUE GOES HERE


%%%% Main document starts here.

\begin{document}

\section*{2.5-2.6}

\begin{description}
    %applied arithmetic and egyptian math
\item[Enlightening Summary \#1:] The Babylonians used a sexagesimal system which allowed them to calculate with fractions with the same facility as integers. The Babylonians also compiled numerous arithmetic tables example squares cubes, square roots, cube roots of numbers 1-50.
The Babylonians were also aware of the quadratic formula for solving quadratic equations, the text describes this discovery through exploring the
relationship between the perimeter and area of a rectangle. The Babylonians also found it absurd that an equation could have to solution, this led to them discarding the negative solutions
from the quadratic formula.  



\item[Enlightening Summary \#2:] With the discovery of the Plimpton 322 we know that the Babylonians discovered the Pythagorean Theorem more than a thousand years before
Pythagoras was born. through analyzing the Plimpton 322 it was also discovered the the babylonians likely used,
\begin{equation*}
    z^2 = (m+n)^2 - 2mn
\end{equation*}
rather than,
\begin{equation*}
    z^2 = m^2 + n^2.
\end{equation*}

\item[Interesting:] I had never heard the story of how the Babylonians discovered the Pythagorean Theorem before Pythagoras. 



\item[Confusing:] On page 70 there is an example of a Babylonian problem that actually contains a negative number, and it specifically
notes that it is in fact a negative number and not symbol for subtraction. So it seems like they were aware of negative numbers, but they also 
refused to acknowledge negative solutions to quadratic equations.


\end{description}
\end{document}
