%%%%%%%%%%%%%%%%%%%%%%%%%%%%%%%%%%%%%%%%%%%%%%%%%%%%%%%%%%%%%%%%%%%%%%%%%%%%%%%%%%%%%%%
%%%%%%%%%%%%%%%%%%%%%%%%%%%%%%%%%%%%%%%%%%%%%%%%%%%%%%%%%%%%%%%%%%%%%%%%%%%%%%%%%%%%%%%
% 
% This top part of the document is called the 'preamble'.  Modify it with caution!
%
% The real document starts below where it says 'The main document starts here'.

\documentclass[12pt]{article}

\usepackage{amssymb,amsmath,amsthm}
\usepackage[top=1in, bottom=1in, left=1.25in, right=1.25in]{geometry}
\usepackage{fancyhdr}
\usepackage{enumerate}
\usepackage{hieroglf}

% Comment the following line to use TeX's default font of Computer Modern.
\usepackage{times,txfonts}

\newtheoremstyle{homework}% name of the style to be used
  {18pt}% measure of space to leave above the theorem. E.g.: 3pt
  {12pt}% measure of space to leave below the theorem. E.g.: 3pt
  {}% name of font to use in the body of the theorem
  {}% measure of space to indent
  {\bfseries}% name of head font
  {:}% punctuation between head and body
  {2ex}% space after theorem head; " " = normal interword space
  {}% Manually specify head
\theoremstyle{homework} 

% Set up an Exercise environment and a Solution label.
\newtheorem*{exercisecore}{Exercise \@currentlabel}
\newenvironment{exercise}[1]
{\def\@currentlabel{#1}\exercisecore}
{\endexercisecore}

\newcommand{\localhead}[1]{\par\smallskip\noindent\textbf{#1}\nobreak\\}%
\newcommand\solution{\localhead{Solution:}}

%%%%%%%%%%%%%%%%%%%%%%%%%%%%%%%%%%%%%%%%%%%%%%%%%%%%%%%%%%%%%%%%%%%%%%%%
%
% Stuff for getting the name/document date/title across the header
\makeatletter
\RequirePackage{fancyhdr}
\pagestyle{fancy}
\fancyfoot[C]{\ifnum \value{page} > 1\relax\thepage\fi}
\fancyhead[L]{\ifx\@doclabel\@empty\else\@doclabel\fi}
\fancyhead[C]{\ifx\@docdate\@empty\else\@docdate\fi}
\fancyhead[R]{\ifx\@docauthor\@empty\else\@docauthor\fi}
\headheight 15pt

\def\doclabel#1{\gdef\@doclabel{#1}}
\doclabel{Use {\tt\textbackslash doclabel\{MY LABEL\}}.}
\def\docdate#1{\gdef\@docdate{#1}}
\docdate{Use {\tt\textbackslash docdate\{MY DATE\}}.}
\def\docauthor#1{\gdef\@docauthor{#1}}
\docauthor{Use {\tt\textbackslash docauthor\{MY NAME\}}.}
\makeatother

% Shortcuts for blackboard bold number sets (reals, integers, etc.)
\newcommand{\Reals}{\ensuremath{\mathbb R}}
\newcommand{\IRats}{\ensuremath{\mathbb I}}
\newcommand{\Nats}{\ensuremath{\mathbb N}}
\newcommand{\Ints}{\ensuremath{\mathbb Z}}
\newcommand{\Rats}{\ensuremath{\mathbb Q}}
\newcommand{\Cplx}{\ensuremath{\mathbb C}}
%% Some equivalents that some people may prefer.
\let\RR\Reals
\let\NN\Nats
\let\II\Ints
\let\CC\Cplx


\doclabel{Math F316}
\docauthor{Stefano Fochesatto}
\docdate{\today}% DATE READING SENTENCES ARE DUE GOES HERE


%%%% Main document starts here.

\begin{document}

\section*{5.3-5.5 Chinese Mathematics}

\begin{description}

\item[Enlightening Summary \#1:] Our reading into the History of Chinese Mathematics begins with outlining the shared knowledge
between the Hindu and Chinese. The section in 5.3 describes the methods the Chinese used to solve indeterminate systems 
of linear and quadratic equations.\\

\item[Enlightening Summary \#2:] Section 5.5 more into detail on the state of Chinese Mathematics. Throughout the section it 
is emphasized that Chinese math was profoundly algebraic and applied, in some cases the antithesis of the Greeks. We are introduced to the
The Nine Chapters of Mathematical Art, which is touted as the oldest textbook on arithmetic in existence. The book's intended use was for 
surveyors and engineers with problems that focused on finding areas, volumes, solving systems of equations, and rules for calculating interest.
We saw another sweet proof for approximating $\pi$, and what looks to be the beginnings of linear algebra in an example where a matrix is reduced to row echelon form.
The section ends with innovations in counting rod notation, binomial expansion, Horner's Method, and the introduction of western mathematics aided by the Jesuit missionary.\\   


\item[Interesting:] I was really surprised when I saw Horner's Method was discovered so early on especially because centuries later its still 
the optimal way of evaluating a polynomial.  


\item[Confusing:] Its seems from the example on page 257 that the Chinese were solving linear systems in the form of a matrix. I feel like solving systems like this would eventually lead to a situataion where 
you have simplified a row to all zeros which I think might lead to the infinite solutions conclusion. However in the section 
before it states that Sun-Tsu might not have been aware that there were infinitely many solutions to indeterminate problems. Is it possible that maybe they didn't 
fully understand the properties of zero?  


\end{description}
\end{document}
