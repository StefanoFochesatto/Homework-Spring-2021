%%%%%%%%%%%%%%%%%%%%%%%%%%%%%%%%%%%%%%%%%%%%%%%%%%%%%%%%%%%%%%%%%%%%%%%%%%%%%%%%%%%%%%%
%%%%%%%%%%%%%%%%%%%%%%%%%%%%%%%%%%%%%%%%%%%%%%%%%%%%%%%%%%%%%%%%%%%%%%%%%%%%%%%%%%%%%%%
% 
% This top part of the document is called the 'preamble'.  Modify it with caution!
%
% The real document starts below where it says 'The main document starts here'.

\documentclass[12pt]{article}

\usepackage{amssymb,amsmath,amsthm}
\usepackage[top=1in, bottom=1in, left=1.25in, right=1.25in]{geometry}
\usepackage{fancyhdr}
\usepackage{enumerate}
\usepackage{hieroglf}

% Comment the following line to use TeX's default font of Computer Modern.
\usepackage{times,txfonts}

\newtheoremstyle{homework}% name of the style to be used
  {18pt}% measure of space to leave above the theorem. E.g.: 3pt
  {12pt}% measure of space to leave below the theorem. E.g.: 3pt
  {}% name of font to use in the body of the theorem
  {}% measure of space to indent
  {\bfseries}% name of head font
  {:}% punctuation between head and body
  {2ex}% space after theorem head; " " = normal interword space
  {}% Manually specify head
\theoremstyle{homework} 

% Set up an Exercise environment and a Solution label.
\newtheorem*{exercisecore}{Exercise \@currentlabel}
\newenvironment{exercise}[1]
{\def\@currentlabel{#1}\exercisecore}
{\endexercisecore}

\newcommand{\localhead}[1]{\par\smallskip\noindent\textbf{#1}\nobreak\\}%
\newcommand\solution{\localhead{Solution:}}

%%%%%%%%%%%%%%%%%%%%%%%%%%%%%%%%%%%%%%%%%%%%%%%%%%%%%%%%%%%%%%%%%%%%%%%%
%
% Stuff for getting the name/document date/title across the header
\makeatletter
\RequirePackage{fancyhdr}
\pagestyle{fancy}
\fancyfoot[C]{\ifnum \value{page} > 1\relax\thepage\fi}
\fancyhead[L]{\ifx\@doclabel\@empty\else\@doclabel\fi}
\fancyhead[C]{\ifx\@docdate\@empty\else\@docdate\fi}
\fancyhead[R]{\ifx\@docauthor\@empty\else\@docauthor\fi}
\headheight 15pt

\def\doclabel#1{\gdef\@doclabel{#1}}
\doclabel{Use {\tt\textbackslash doclabel\{MY LABEL\}}.}
\def\docdate#1{\gdef\@docdate{#1}}
\docdate{Use {\tt\textbackslash docdate\{MY DATE\}}.}
\def\docauthor#1{\gdef\@docauthor{#1}}
\docauthor{Use {\tt\textbackslash docauthor\{MY NAME\}}.}
\makeatother

% Shortcuts for blackboard bold number sets (reals, integers, etc.)
\newcommand{\Reals}{\ensuremath{\mathbb R}}
\newcommand{\IRats}{\ensuremath{\mathbb I}}
\newcommand{\Nats}{\ensuremath{\mathbb N}}
\newcommand{\Ints}{\ensuremath{\mathbb Z}}
\newcommand{\Rats}{\ensuremath{\mathbb Q}}
\newcommand{\Cplx}{\ensuremath{\mathbb C}}
%% Some equivalents that some people may prefer.
\let\RR\Reals
\let\NN\Nats
\let\II\Ints
\let\CC\Cplx


\doclabel{Math F316}
\docauthor{Stefano Fochesatto}
\docdate{\today}% DATE READING SENTENCES ARE DUE GOES HERE


%%%% Main document starts here.

\begin{document}

\section*{2.1-2.4}

\begin{description}
    %applied arithmetic and egyptian math
\item[Enlightening summary \#1:] Section 2.3 discusses the how the Rhind Papyrus contains several
arithmetic problems that are solved using the method of simple false position and double false position.
Interestingly double false position is just another form of linear interpolation, as we can derive the formula on page $48$
by simply looking at the point slope form of a line from the two guesses and finding the root. Suppose we want to solve $ax+b = 0$, with 
the two guesses being  $(g_1, f_1)$ and $(g_2, f_2)$. Consider the line through those points in point-slope form, 
\begin{equation*}
    y - f_1 = \dfrac{f_1-f_2}{g_1 - g_2}(x - g_1).
\end{equation*}
Let $y = 0$ and solve for $x$, 
\begin{align*}
    -f_1 &= \dfrac{f_1-f_2}{g_1 - g_2}(x - g_1),\\
    -f_1\dfrac{g_1 - g_2}{f_1-f_2} &= x - g_1,\\
    g_1 - f_1\dfrac{g_1 - g_2}{f_1-f_2} &= x,\\
    \dfrac{g_1(f_1-f_2)}{f_1-f_2} - \dfrac{f_1(g_1 - g_2)}{f_1-f_2} &= x,\\
    \dfrac{f_1g_2 - f_2g_1}{f_1-f_2} &= x.\\
\end{align*}
Although the method of simple false position has become more obscure, double false position forms the basis of the Regula Falsi 
root finding method, which still holds some relevancy in numerical analysis.


%applied geometry and the theories regarding the great
\item[Enlightening summary \#2:] Section 2.4 discusses the origins of geometry and how ancient egyptians were able to approximate various 
geometric formulas like, the area of a circle, trapezoid, and volume of a truncated pyramid. Beyond that the rest of this section is centered around dispelling myths 
about the Great Pyramid with the underlying message that egyptian geometry never advanced to a stage where rigor was at the forefront. 


\item[Interesting:] I found the story of the Rosetta Stone, and the Rhind Papyrus to be very engrossing. I wouldn't be surprised if there was ever a film made about Jean Francois 
Champollion and his efforts in deciphering egyptian Hieroglyphics. Also the story of how the Rhind Papyrus was pieced together seemed eerily preordained. 

\item[Confusing:] In general I found the method of representing rational numbers in section 2.2 confusing. On page 45 we are given a proof for why the method of splitting a rational function into unit function terminates. 
The proof uses the monotone convergence theorem without explicitly stating it. 


\end{description}
\end{document}
