%%%%%%%%%%%%%%%%%%%%%%%%%%%%%%%%%%%%%%%%%%%%%%%%%%%%%%%%%%%%%%%%%%%%%%%%%%%%%%%%%%%%%%%
%%%%%%%%%%%%%%%%%%%%%%%%%%%%%%%%%%%%%%%%%%%%%%%%%%%%%%%%%%%%%%%%%%%%%%%%%%%%%%%%%%%%%%%
% 
% This top part of the document is called the 'preamble'.  Modify it with caution!
%
% The real document starts below where it says 'The main document starts here'.

\documentclass[12pt]{article}

\usepackage{amssymb,amsmath,amsthm}
\usepackage[top=1in, bottom=1in, left=1.25in, right=1.25in]{geometry}
\usepackage{fancyhdr}
\usepackage{enumerate}
\usepackage{hieroglf}

% Comment the following line to use TeX's default font of Computer Modern.
\usepackage{times,txfonts}

\newtheoremstyle{homework}% name of the style to be used
  {18pt}% measure of space to leave above the theorem. E.g.: 3pt
  {12pt}% measure of space to leave below the theorem. E.g.: 3pt
  {}% name of font to use in the body of the theorem
  {}% measure of space to indent
  {\bfseries}% name of head font
  {:}% punctuation between head and body
  {2ex}% space after theorem head; " " = normal interword space
  {}% Manually specify head
\theoremstyle{homework} 

% Set up an Exercise environment and a Solution label.
\newtheorem*{exercisecore}{Exercise \@currentlabel}
\newenvironment{exercise}[1]
{\def\@currentlabel{#1}\exercisecore}
{\endexercisecore}

\newcommand{\localhead}[1]{\par\smallskip\noindent\textbf{#1}\nobreak\\}%
\newcommand\solution{\localhead{Solution:}}

%%%%%%%%%%%%%%%%%%%%%%%%%%%%%%%%%%%%%%%%%%%%%%%%%%%%%%%%%%%%%%%%%%%%%%%%
%
% Stuff for getting the name/document date/title across the header
\makeatletter
\RequirePackage{fancyhdr}
\pagestyle{fancy}
\fancyfoot[C]{\ifnum \value{page} > 1\relax\thepage\fi}
\fancyhead[L]{\ifx\@doclabel\@empty\else\@doclabel\fi}
\fancyhead[C]{\ifx\@docdate\@empty\else\@docdate\fi}
\fancyhead[R]{\ifx\@docauthor\@empty\else\@docauthor\fi}
\headheight 15pt

\def\doclabel#1{\gdef\@doclabel{#1}}
\doclabel{Use {\tt\textbackslash doclabel\{MY LABEL\}}.}
\def\docdate#1{\gdef\@docdate{#1}}
\docdate{Use {\tt\textbackslash docdate\{MY DATE\}}.}
\def\docauthor#1{\gdef\@docauthor{#1}}
\docauthor{Use {\tt\textbackslash docauthor\{MY NAME\}}.}
\makeatother

% Shortcuts for blackboard bold number sets (reals, integers, etc.)
\newcommand{\Reals}{\ensuremath{\mathbb R}}
\newcommand{\IRats}{\ensuremath{\mathbb I}}
\newcommand{\Nats}{\ensuremath{\mathbb N}}
\newcommand{\Ints}{\ensuremath{\mathbb Z}}
\newcommand{\Rats}{\ensuremath{\mathbb Q}}
\newcommand{\Cplx}{\ensuremath{\mathbb C}}
%% Some equivalents that some people may prefer.
\let\RR\Reals
\let\NN\Nats
\let\II\Ints
\let\CC\Cplx


\doclabel{Math F316}
\docauthor{Stefano Fochesatto}
\docdate{\today}% DATE READING SENTENCES ARE DUE GOES HERE


%%%% Main document starts here.

\begin{document}

\section*{3.1-2.3}

\begin{description}
%
\item[Enlightening summary \#1:] Section 3.1 introduces us to Thales of Miletos and gives a broad overview of how Grecian 
mathematics was more abstract and rational than that of the Egyptians or Babylonians. The text mainly attributes this growth to 
a phenomenon called the "Greek Miracle" wherein the rapid expansion of the Greek empire, led to pluralism between the older Egyptian and
Babylonian cultures which produced the majority of early greek mathematics.
%
\item[Enlightening summary \#2:]  Section 3.2 introduces us to Pythagoras and his school of followers. The text outlines 
their discovery of some number theory theorems, such as a closed form for a finite series of odd and even integers. Section 2.3 focuses on 
Pythagoras' solutions to the pythagoras equation, and the many ways incommensurable quantities were approximated. 


\item[Interesting:] I was surprised at the way that Theon of Smyrna was able to approximate irrational numbers. The text gives a proof for why 
Theon's process converges to a value of $\sqrt{2}$ however it does not describe the motivation for his process, beyond labeling certain values as side and diagonal.   
I also really enjoyed the geometric argument that comes afterward from "Euclid's Elements" for why $\sqrt{2}$ is irrational. I also found Pythagoras' aversion to beans very intriguing.

\item[Confusing:] A lot of times the text continues an idea that the Greeks had, without clarifying if it was known in their time. This happens when it shows that Theon's process for approximating $\sqrt{2}$
applies to all irrationals of the form $\sqrt{a}$, $a \in \Nats$. The same thing is true for 
the proof of the sum of the first $n^2$ numbers. Also it seems like the Greeks were aware of various proof techniques, like contradiction, contrapositive, and induction but 
its not really discussed in the text.

\end{description}
\end{document}
